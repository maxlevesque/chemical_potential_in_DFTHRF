\documentclass[aps,pre,preprint,showpacs,amsmath,amssymb,superscriptaddress]{revtex4-1}

%\usepackage{calc}
\usepackage{graphicx}
\usepackage{bm}
%\usepackage{color}

%~ \makeatletter

%~ \makeatother
\newcommand{\rr}{\mathbf{r}}
\newcommand{\rrp}{\mathbf{r}^\prime}
\newcommand{\rhor}{\rho\left(\mathbf{\rr}\right)}
\newcommand{\rhorp}{\rho\left(\mathbf{\rrp}\right)}



\begin{document}

\title{Thoughts about the definition of the chemical potential in DFT-HRF}


\author{Maximilien Levesque}
\email{maximilien.levesque@gmail.com}
\affiliation{UPMC Univ Paris 06, UMR 7195, PECSA, F-75005, Paris, France}
\affiliation{CNRS, UMR 7195, PECSA, F-75005, Paris, France}


\begin{abstract}
In the homogeneous reference fluid approximation one may use in the classical version of the density functional theory,
what exactly is the chemical potential one imposes?
It is actually the difference between the chemical potential of the homogeneous fluid, and an ideal gas the
same density as the liquid.
\end{abstract}

\maketitle

\section{Theory}

The free energy functional of an homogeneous fluid of particles interacting with pair interactions only and density $\rho_0$ is expressed as
\begin{equation}
    \beta F[\rhor] = \int \left( \rhor \ln \frac{\rhor}{\rho_0} - \rhor \right) d\rr +\frac{1}{2}\int d\rr \int d\rrp c(\rr,\rrp) \rhor \rhorp,
\end{equation}
where $\beta=1/k_B T$, and $c(\rr,\rrp)$ is the pair correlation function of the fluid.

The chemical potential $\mu$ of the fluid is defined in MDFT-HRF by
\begin{equation}
    \beta\mu = \frac{\partial F[\rhor]}{\partial \rhor}\Big|_{\rhor=\rho_0},
\end{equation}
which after trivial developments lead to
\begin{equation}
    \beta\mu = \ln\frac{\rho_0}{\rho_0} + B\rho_0 = B\rho_0,
\end{equation}
where $B$ is linked to the virial coefficients (which one? ;)).


\begin{acknowledgments}
ML thanks Dr Benjamin Rotenberg for bringing out the need for such paper and for fruitful discussions.
\end{acknowledgments}

\bibliographystyle{apsrev4-1}
\bibliography{lb}

\end{document}
